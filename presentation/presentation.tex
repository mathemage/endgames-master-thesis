\documentclass{beamer}
\usetheme{metropolis}           % Use metropolis theme
\usepackage{appendixnumberbeamer}
\usepackage{epigraph}
\usepackage{color}
\usepackage{amsopn}
\usepackage{tabto}

%%% Bibliography
\usepackage[backend=bibtex, style=authoryear]{biblatex}
\AtBeginBibliography{\tiny}
\bibliography{../src/bibliography.bib}

%\setbeamercolor{background canvas}{bg=white}
%\setbeamercolor{title}{fg=white}
%\setbeamercolor{subtitle}{fg=black}
%\setbeamercolor{author}{fg=black}
%\setbeamercolor{institute}{fg=white}

\newcommand{\todo}{\alert{TODO}}
\newcommand{\itemBullet}{\scriptsize$\blacksquare$}
\setbeamertemplate{itemize item}{\itemBullet}
\setbeamertemplate{itemize subitem}{\itemBullet}
\setbeamertemplate{itemize subsubitem}{\itemBullet}
\newcommand{\E}{\mathop{\mathbb{E}}}
\DeclareMathOperator*{\argmax}{arg\,max}
\newcommand{\epiParSpace}{\vskip 1.5ex}

\title{Solving Endgames \\in~Large Imperfect-Information Games \\such as~Poker}
\date{September 13, 2016}
\author{Bc.~Karel Ha}
\institute{Department of~Applied Mathematics \\Charles University}

\begin{document}
  \maketitle

  \section{Perfect-Information Subgames}
  % todo Lomonosov chess
  % todo Go CGT
  % todo Go AlphaGo

  \section{Imperfect-Information Subgames}
  % todo df: (imperf-info) subgame
  % todo solving games: sequences, sequence-form LP
  % todo solving games: learning algos

  \section{Endgame Solving}
  % todo gadget
  % todo LP

  \section{CFR-D Decomposition}
  % todo gadget
  % todo LP

  \section{Subgame-Margin Maximization}
  % todo df: margin
  % todo thm: improvement-propto-margin
  % todo LP
  % todo gadget

  \section{Ideas for Future Work}
  % todo VLP

%%%%%%%%%%%%%%%%%%%%%%%%%%%%%%%%%%%%%%%%%%%%%%%%%%%%%%%%%%%%%%%%%%%%%%%%%%%%%%%%

  \begin{frame}[allowframebreaks]{References}
    \tiny
    \printbibliography[heading=none]
  \end{frame}

\end{document}
