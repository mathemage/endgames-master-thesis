\chapter{Background, Definitions and Notations}

The field of Game Theory deals with interactions or conflicts between $2$ or more agents.
\todo

\section{General Game Theory}

\begin{itemize}
  \item players
  \item actions
  \item payoffs
\end{itemize}

\subsection{Extensive form}

An \emph{extensive-form game} consists of

\begin{itemize}
  \item a finite set of \emph{players}
  \item a finite set $H$ of all possible states or also \emph{histories}.
    \begin{itemize}
      \item \emph{action}
      \item \emph{terminal history}
    \end{itemize}
  \item $A(h) = $
  \item a~function $p$ assigning an~\emph{acting player} to \todo.
    It forms the set $P \cup c$, where \todo
  \item a~function $f_c$ determining the probability measure on $A(h)$ for every node $h$ where $p(h) = c$.
    That is, the nodes of the chance player.
  \item \emph{information partition}. \emph{information set}
  \item \emph{utility function}
\end{itemize}

There may be various possible properties for extensive-form games.
An~extensive-form game can:

\begin{itemize}
  \item be \emph{two-player}
  \item have \emph{perfect recall}
  \item be \emph{zero-sum}
\end{itemize}

\subsection{Sequence form}

\section{Methods of~solution}

\subsection{Linear programming}

\subsection{Learning}

\section{Subgames (endgames)}

\subsection{Previous works}
