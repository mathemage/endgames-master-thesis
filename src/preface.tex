\chapter*{Introduction}
\addcontentsline{toc}{chapter}{Introduction}
\epigraphLong{
  A~successful person isn't necessarily better than her less successful peers at~solving problems;
  her pattern-recognition facilities have just learned what problems are worth solving.
}{Ray Kurzweil}
Endgames have a~special role for game playing:
the situation towards the ending becomes simpler and many game aspects are already clear, e.g., remaining late-game pieces in~Chess or the division of~territories in~Go.
Endgames therefore offer the opportunity for a~deeper analysis, more than it would be possible during an~opening or a~mid-game.
Additionally, they make the exhaustive search possible.

Such an~approach is fruitful in~games with perfect information (Part I), where it has been used extensively:
the perfect play has been pre-computed for Chess endgames with up to 7 pieces (Chapter~\ref{ch:Chess}), while Go endgames may undergo dynamical in-play analysis, either using \acrlong{cgt} (Chapter~\ref{ch:goCGT}) or Monte Carlo simulations aided by deep \acrlong{cnn}s (Chapter~\ref{ch:AlphaGo}).

An~appealing idea is to extend this approach to imperfect-information games:
to play the early parts of~the game, and once the subgame becomes tractable, to calculate a~solution using a~finer-grained abstraction in~real time, creating a~combined final strategy.
Although this approach is straightforward for perfect-information games, it is a~much more complex problem for imperfect-information games (Part II).

Firstly, we have to take information sets into consideration, and generalize the notion of~a~\emph{subgame} for the imperfect information (Chapter~\ref{ch:imperf-intro}).
Unfortunately, directly re-solving a~subgame does not in general preserve any guarantee of~optimality, leading to far more exploitable combination of~resulting strategies.
We will see this on an~elementary example of~\acrlong{rps} and its only subgame (Chapter~\ref{ch:imperf-shortcomings}).

The posed challenge may be tackled in three ways:
\begin{enumerate}[(a)]
  \item Ganzfried and Sandholm (Carnegie Mellon University) disregard the problem entirely, by advocating empirical improvement over theoretical one (Chapter~\ref{ch:endgame-solving}).
  \item Burch, Johanson, and Bowling (University of~Alberta) propose a~decomposition technique, which retains at~least the same quality of~subgame strategies.
    Nonetheless, the method receives no incentive to strive for maximal improvement (Chapter~\ref{ch:cfr-d}).
  \item We---Moravčík, Schmid, Ha, Hladík (Charles University) and Gaukrodger  (Koypetition)---formalize the improvement of~subgame strategies into the notion of~a~\emph{\acrlong{sm}}.
    On top of~that, we devise a~construction of~an~equivalent \acrlong{efg}, which manages to maximize \acrlong{sm}s (Chapter~\ref{ch:max-margin}).
\end{enumerate}
