\chapter*{Introduction}
\addcontentsline{toc}{chapter}{Introduction}
\epigraphLong{
  A~successful person isn't necessarily better than her less successful peers at~solving problems;
  her pattern-recognition facilities have just learned what problems are worth solving.
}{Ray Kurzweil}
Endgames have a~special role for game playing:
the situation towards the ending becomes simpler and many game aspects are already clear, e.g., remaining late-game pieces in~Chess or the division of~territories in~Go.
Endgames therefore offer the opportunity for a~deeper analysis, more than it would be possible during an~opening or a~mid-game.
Additionally, they make the exhaustive search possible.

Such an~approach is fruitful in~games with perfect information, where it has been used extensively:
the perfect play of~Chess endgames has been pre-computed for up to 7 pieces (Chapter~\ref{ch:Chess}), while Go endgames may undergo dynamical in-play analysis, either using \acrlong{cgt} (Chapter~\ref{ch:goCGT}) or Monte Carlo simulations aided by deep convolutional neural networks (Chapter~\ref{ch:AlphaGo}).

An~appealing idea is to extend this approach to imperfect-information games:
to play the early parts of~the game and then, once the subgame becomes tractable, to calculate a~solution using a~finer-grained abstraction in real time, creating a~combined final strategy.
Although this approach is straightforward for perfect-information games, it is a~much more complex problem for imperfect-information games. 
