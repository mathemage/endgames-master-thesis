\chapter*{Introduction}
\addcontentsline{toc}{chapter}{Introduction}
\epigraphLong{
  A~successful person isn't necessarily better than her less successful peers at~solving problems;
  her pattern-recognition facilities have just learned what problems are worth solving.
}{Ray Kurzweil}
Endgames have a~special role in~games:
the situation towards the ending becomes simpler and many game aspects are already clear, e.g., division of~territories in~Go or remaining pieces in~Chess.
Endgames therefore offer the opportunity to analyze the game exhaustively, more than would be possible during an~opening or a~mid-game.
Such an~approach is fruitful in~games with perfect information and has been used extensively:
the perfect play of~Chess endgames has been pre-computed for up to 7 pieces, while Go endgames may undergo dynamical in-play analysis, either using \acrlong{cgt} or Monte Carlo simulations.

An~appealing idea to solve endgames is to play the early parts of~the game and then, once the subgame becomes tractable, to calculate a~solution using a~finer-grained abstraction in real time, creating a~combined final strategy.
While this approach is straightforward for perfect-information games, it is a~much more complex problem for imperfect-information games. 

