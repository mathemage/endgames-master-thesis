\chapter*{Introduction}
\addcontentsline{toc}{chapter}{Introduction}
\epigraphLong{
  A~successful person isn't necessarily better than her less successful peers at~solving problems;
  her pattern-recognition facilities have just learned what problems are worth solving.
}{Ray Kurzweil}
Endgames have a~special role for game playing:
the situation towards the end becomes simpler and many game aspects are clearly defined, e.g., remaining late-game pieces in~Chess, or the division of~territories in~Go.
As~opposed to~openings and mid-games, endgames therefore offer the opportunity for a~thorough scrutiny such as an~exhaustive search.

This approach is fruitful in~games with perfect information (Part I), where there has been an extensive use of~it:
a~perfect play has been pre-computed for Chess endgames with up to 7 pieces (Chapter~\ref{ch:Chess}), while Go endgames can undergo dynamical in-play analysis, either using \acrlong{cgt} (Chapter~\ref{ch:goCGT}) or Monte Carlo simulations aided by \acrlong{cnn}s (Chapter~\ref{ch:AlphaGo}).

An~appealing idea is to extend this approach to imperfect-information games:
play the early parts of~the game, and once the subgame becomes feasible, calculate a~solution using a~finer-grained abstraction in~real time, creating a~combined final strategy.
Although this approach is straightforward for perfect-information games (Chapter~\ref{ch:perf-intro}), it is a~much more complex problem for imperfect-information games (Part~II).

Firstly, we have to take into consideration \emph{information sets}, and generalize the concept of~a~\emph{subgame} for imperfect information (Chapter~\ref{ch:imperf-intro}).
Unfortunately, directly re-solving a~subgame does not in general preserve any guarantee of~optimality, resulting in~far more exploitable strategies.
We will observe this phenomenon on~the elementary example of~\acrlong{rps} game with a~single subgame (Chapter~\ref{ch:imperf-shortcomings}).

The posed challenge may be tackled in three different ways:
\begin{enumerate}[(a)]
  \item Ganzfried and Sandholm (Carnegie Mellon University) disregard the problem entirely, by advocating empirical improvement over theoretical guarantees (Chapter~\ref{ch:endgame-solving}).
  \item Burch, Johanson, and Bowling (University of~Alberta) propose a~decomposition technique, which retains at~least the same quality of~subgame strategies.
    Nonetheless, the method receives no incentive to strive for maximal gains (Chapter~\ref{ch:cfr-d}).
  \item We---Moravčík, Schmid, Ha, Hladík (Charles University) and Gaukrodger  (Koypetition)---formalize the improvement of~subgame strategies into the notion of~a~\emph{\acrlong{sm}}.
    On top of~that, we devise a~construction of~an~equivalent \acrlong{efg}, which maximizes \acrlong{sm}s (Chapter~\ref{ch:max-margin}).
\end{enumerate}

Finally, Chapter~\ref{ch:future-work} provides some suggestions to further develop these ideas, and enhance them for future work.
