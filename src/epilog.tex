\chapter*{Conclusion}
\addcontentsline{toc}{chapter}{Conclusion}
\epigraphLong{
  Everything that civilisation has to~offer is a~product of human intelligence;
  we cannot predict what we might achieve when this intelligence is magnified by~the tools that AI may provide, but the eradication of~war, disease, and poverty would be high on~anyone's list.
  Success in~creating AI would be the biggest event in~human history.
  Unfortunately, it might also be the last.
}{Stephen Hawking}
Endgames have special role in~games:
the situation towards the end becomes simpler and many game aspects are already clear, e.g., division of~territories in~Go or remaining pieces in~Chess.
Endgames therefore offer the opportunity to analyze the game exhaustively, more than would be possible during an~opening or a~mid-game.
Such an~approach is fruitful in~games with perfect information and has been used extensively:
the perfect play of~Chess endgames has been pre-computed for up to 7 pieces, while Go endgames may undergo dynamical in-play analysis, either using \acrlong{cgt} or Monte Carlo simulations.

For the imperfect information, on~the other hand, endgames need to be adjusted in~order to account for information sets.
This gives rise to a~new definition of~an~(imperfect-information) \emph{subgame}.
As such, the definition does not directly allow to apply the same procedure from perfect-information endgames:
even under the simplest conditions of~the \acrlong{rps} game, a~na{\"i}ve endgame re-solution fails to form a~\acrlong{ne}.

This occurs because the opponent can change his behavior prior to the endgame.
Such an~exploitation power can be captured by~a~\emph{counterfactual value}:
a~hypothetical ``what-if'' value summarizing opponent's improvement, if he had changed his prior trunk strategy.

We used counterfactual values to define our own notion of~\emph{\acrlong{sm}}:
a~gap between the original and the new \acrlong{cbv}s.
We related the \acrshort{sm} to the exploitability against a~\acrlong{br}, and we proved the overall improvement rising from endgame improvement is proportional to the \acrshort{sm}.

Maximizing \acrshort{sm} is thus highly advisable.
That can be achieved either by solving a~variant of a~sequence-form \acrlong{lp}, or by applying an~iterative learning algorithm to a~\emph{gadget game}, an~equivalent ad-hoc \acrlong{efg}.
The latter approach offers greater benefits in~terms of~exploiting domain-specific knowledge and employing powerful learning algorithms such as \acrshort{cfr}, \acrshort{mccfr} or the~modern \cfrplus{} that we chose to solve our \emph{max-margin gadget}.

Finally, we experimentally compared the following approaches:
\begin{enumerate}[(i)]
  \item endgame solving (\cite{Ganzfried2015endgame})
  \item \acrshort{cfr-d} and decomposition (\cite{BurchJohansonBowling2014})
  \item our subgame-margin maximization (\cite{Moravcik2016refining})
\end{enumerate}
The (i) even produced a~worse \acrshort{sm}, leading to more exploitable subgame strategies;
and although the \acrshort{sm} of~(ii) increased over time, the improvement was not significant, as the method guarantees only the same quality, not the best one.
Hence our (iii), specially designed to maximize \acrshort{sm}s, proved to re-create the most robust (i.e., the least exploitable) subgame strategy.
We thus offer a~superior solution to treating subgames.
