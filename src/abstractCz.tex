%%% for abstract in Czech
%\selectlanguage{czech}
\thispagestyle{empty}
\openright

\vbox to 0.5\vsize{
\setlength\parindent{0mm}
\setlength\parskip{5mm}

Název práce:
Řešení koncovek ve velkých hrách s neúplnou informací jako je např. Poker

Autor:
\ThesisAuthor

Katedra:
Katedra aplikované matematiky

Vedoucí diplomové práce:
\Supervisor, Katedra aplikované matematiky

Abstrakt:
Koncovky mají významnou roli pro hráče.
V pozdních fázích hry je mnoho aspektů již jasně definováno, což mnohdy umožňuje rozebrat všechny možnosti.
Speciální zacházení s~koncovkami je obzvláště účinné pro hry s~úplnou informací, např. databáze šachových koncovek předvyřešené pro celé třídy typů zakončení, anebo v~Go rozdělení desky do samostatných nezávislých podher.

Je lákavé rozšířit tento přístup i na~hry s~neúplnou informací, jakým je např. známý Poker.
Zahrát počáteční fáze hry, a jakmile podhry začnou být zvlád\-nu\-tel\-né, vypočítat pro ně koncové řešení zvlášť.
Ovšem situace je mnohem komplikovanější pro hry s~neúplnou informací.

\emph{Podhry} je potřeba zobecnit pro neúplnou informaci kvůli \emph{informačním množinám}.
Bohužel takové zobecnění nelze hned řešit přímo, neboť by nebyla zachována optimalita.
V důsledku toho můžeme skončit s mnohem ovlivnitelnější strategií (co se týče zneužitelnosti).

V současnosti jsou tři přístupy, jak se s~touto výzvou vypořádat:
\begin{enumerate}[(a)]
  \item tento problém zcela zanedbat,
  \item použít techniku podrozdělení, která však většinou zachovává jenom stejně dobrá řešení,
  \item nebo zformalizovat pojem \emph{rozmezí podhry} jakožtu míra zlepšení strategie.
    Pro ní zkonstruujeme pomocnou hru, jež rozmezí podhry optimalizuje.
\end{enumerate}

Poslední přístup je náš vlastní výsledek, který jsme v~roce~2016 prezentovali na~třicáté konferenci AAAI pro umělou inteligenci.

Též jsme skrze pokusy porovnali tři zmíněná řešení, pomocí špičkového účastníka soutěže pokerových programů na AAAI-14.
Tato soutěž je předním vývojovým prostředím pro agenty s~neúplnou informací.

Klíčová slova:
{%
  {algoritmická teorie her}, {hry s~neúplnou informací}, {Nashovo equilibrium},
  {podhra}, {koncovka}, {counterfactual regret minimization}, {Poker}
}

\vss}

\newpage
