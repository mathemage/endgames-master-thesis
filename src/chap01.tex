\chapter{Endgames: Perfect Situation in~Perfect-Information Games}

For many games, solving their late stages (so-called \emph{endgames}) can be done in~an~online way.
In other words, we are often able to postpone the computation of~the endgame strategy until the endgame itself is reached in~the play.

This is especially the case of~perfect-information games such as chess or Go, where the endgame technique has been used for long time.
In these domains, online endgame solving has~significant importance, as it substantially improves the playing quality of~agents.

\section{Chess Endgames}
\todo

\section{Go Endgames}
\todo

\subsection{Why Focus on Go Endgames?}

Martin~M{\"u}ller (in~his dissertation~\cite{Muller1995computer}) mentions the following advantages of~Go endgames for research:
\begin{itemize}
  \item The complexity of~the game often decreases towards the end.
    This allows the study of~Go in a~controlled, simplified context.
  \item An~exact solution is possible for some classes of~endgame positions.
  \item The exact solution of~parts of~a~Go board facilitates the analysis of the rest.
    Reaching the ultimate goal of~winning the game is easier when complete information about part of~the game is at~hand.
    Such information is useful as~additional input for the heuristics that deal with the rest of~the board.

    Human experts use similar reasoning: they observe the score continually from the early midgame, and base their strategic decisions on~such an~analysis (\cite{Takagawa85}).
  \item Some methods developed for partitioning, searching and scoring during endgame carry over to the midgame and opening.
    As programs improve, fewer game-deciding blunders will occur, so the importance of~endgame-type calculation is bound to~increase.
  \item On~a~more philosophical note, the endgame relates to the full game of~Go similarly as Go relates to~real world AI problems.
    It provides a~simplified, more controlled sub-domain that allows the use of~stronger theoretical models than the larger, more general problem.
\end{itemize}

\subsection{Partitioning an~Endgame into Local (Sub)games}

A~Go position usually consists of~several local scenes that can be analyzed individually.
In the opening, these scenes can be far apart, and their influence on~each other may be weak.
A~better partition occurs late in the game, when there are walls of~stones dividing the board.
A~move cannot have any influence across a~wall of~safe stones.~(\cite{Muller1995computer})

\subsection{AlphaGo}
\todo
