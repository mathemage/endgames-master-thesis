\chapter{Ideas for Future Work}
\label{ch:future-work}
\epigraphLong{
  The worst thing that can happen in a~democracy---as well as in an~individual's life---is to become cynical about the future and lose hope.
}{Hillary Clinton}
One may try to study other aspects of~endgames.
Some recommendations as inspiration for future work may include
\begin{itemize}
  \item other endgame-specific concepts like \acrlong{sm}
  \item dynamical adjustment of~computation upon entering the ending phase
  \item solving several endgames in~parallel, without interdependence
  \item a~Poker version of~Chess endgame tablebases, which poker agents may use in~simpler ending situations to play perfectly
\end{itemize}

\section{Margins via Multi-Objective Optimization}
\epigraphLong{
  Raise your quality standards as~high as you can live with, avoid wasting your time on~routine problems, and always try to~work as~closely as~possible at~the boundary of~your abilities.
  Do this, because it is the only way of~discovering how that boundary should be moved forward.
}{Edsger Dijkstra}
One especially noteworthy research topic is the study of~margins (i.e., declines of~\acrshort{cbv}s).
The \acrlong{sm} is their $\min$-aggregation.
Tempting idea is to look into alternative aggregation.
Consider a~multi-criteria analogy to~subgame margin maximization (\ref{lp:max-margin}):
\begin{equation}
  \label{vlp:max-margins}
  \begin{split}
    \max_{v, x}\ &\textcolor{red}{\vect{m}} \\
    v_I - \vect{m}_{\textcolor{red}{I}} &\ge CBV_2^{\sigma_1}(I), \quad I \in \I_2^{R(S)}\\ 
    Ex &= e \\
    F^\top v - A_2^\top x &\le \vect{0} \\
    x &\ge \vect{0}.
  \end{split}
\end{equation}
Now $\vect{m}_I$ corresponds to only one margin $\vect{m}_I := CBV_2^{\sigma_1} (I) - CBV_2^{\sigma_1'} (I)$, and $\vect{m} := (m_I) _{I\in\I_2^{R(S)}}$ is a~vector of~all such margins.
Evidently, this \acrfull{vlp} is a~task of~\emph{multi-objective optimization} (\cite{Ehrgott2006multicriteria, Grygarova1996zaklady}).

Notice that (\ref{lp:max-margin}) is just a~special case of~the vector objective function, by~the scalarization
\[
  \max_{v, x}\ \min_{I \in \I_2^{R(S)}}\ {\vect{m}_I}.
\]
The choice of~minimum function stems from the nature of~the opponent, who makes his best to exploit us and hence aims for minimal decrease in~\acrshort{cbv}s.

Undoubtedly, one can also try other methods to deal with multi-objective functions:
\begin{itemize}
  \item other scalarizations, e.g., weighted sum of~margins
  \item (Pareto) efficient solutions
  \item \acrfull{dea}:
    \href{http://www.deazone.com/}{http://www.deazone.com/}
  \item \acrfull{emo}, which uses evolutionary algorithms:
    \href{https://www.cs.cinvestav.mx/~EVOCINV/}{https://www.cs.cinvestav.mx/$\sim$EVOCINV/}
\end{itemize}
It might be compelling as well as intriguing to understand their meaning in~the language of~\acrlong{efg}s, and perhaps even find their corresponding gadget games.
