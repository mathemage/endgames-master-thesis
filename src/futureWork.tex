\chapter{Ideas for Future Work}
\epigraphLong{
  The worst thing that can happen in a~democracy---as well as in an~individual's life---is to become cynical about the future and lose hope.
}{Hillary Clinton}
\todo

\section{Other Possible Objective Functions}
\epigraphLong{
  Raise your quality standards as~high as you can live with, avoid wasting your time on~routine problems, and always try to~work as~closely as~possible at~the boundary of~your abilities.
  Do this, because it is the only way of~discovering how that boundary should be moved forward.
}{Edsger Dijkstra}
Consider a~multi-criteria analogy to~subgame margin maximization (\ref{lp:max-margin}):
\begin{equation}
  \label{vlp:max-margins}
  \begin{split}
    \max_{v, x}\ &\textcolor{red}{\vect{m}} \\
    v_I - \vect{m}_{\textcolor{red}{I}} &\ge CBV_2^{\sigma_1}(I), \quad I \in \I_2^{R(S)}\\ 
    Ex &= e \\
    F^\top v - A_2^\top x &\le \vect{0} \\
    x &\ge \vect{0}.
  \end{split}
\end{equation}
Now $\vect{m}_I$ corresponds to only one margin $\vect{m}_I := CBV_2^{\sigma_1} (I) - CBV_2^{\sigma_1'} (I)$, and $\vect{m} := (m_I) _{I\in\I_2^{R(S)}}$ is a~vector of~all such margins.
Evidently, this \acrfull{vlp} is a~task of~\emph{multi-objective optimization}.

Notice that (\ref{lp:max-margin}) is just a~special case of~the vector objective function, by~the scalarization
\[
  \max_{v, x}\ \min_{I \in \I_2^{R(S)}}\ {\vect{m}_I}.
\]
The choice of~minimum function stems from the nature of~the opponent, who makes his best to exploit us and hence aims for minimal decrease in~\acrshort{cbv}s.

\todo
