\chapter{Subgame-Margin Maximization}
\label{ch:max-margin}
\note{
  This chapter summarizes our own paper~(\cite{Moravcik2016refining}).
}

\section{Motivation}

\section{Subgame Margin}

\begin{defn}[Subgame margin]
\end{defn}

\begin{thm}[improvement is proportional to the subgame margin (and may be greater)]
  \todo
\end{thm}

\begin{proof}
  \todo
\end{proof}

\section{Equivalent Linear Program}
\begin{equation}
  \label{lp:max-margin}
  \begin{split}
    \max_{v, x}\  &m \\
    v_I - m &\ge CBV_2^\sigma(I), \quad I \in \I_2^{R(S)}\\ 
    Ex &= e \\
    F^\top v - A^\top x &\le \vect{0} \\
    x &\ge \vect{0},
  \end{split}
\end{equation}

\section{Gadget Game}
\epigraphLong{
  A~new gadget that lasts only five minutes is worth more than an~immortal work that bores everyone.
}{Francis Picabia}
All states in~the original subgame are directly copied into the resulting gadget game.
We create the gadget game by~making two alterations to~the original subgame:
\begin{enumerate}[(i)]
  \item we shift player~$2$'s utilities using the $CBV_2$ (to initialize all $2$'s values to~zero)
  \item we add $2$'s node followed by chance nodes $c_{I_1}, c_{I_2}, c_{I_3}, \dots$ at~the top of the subgame (to allow the opponent to~pick any starting state, relating the game values to~the margin)
\end{enumerate}
We will distinguish the states, strategies, utilities, etc. for the gadget game by~adding a~tilde to corresponding notations.
The following is a~description of~the steps (see also Figure~\ref{fig:max-margin-gadget} that visualizes the gadget).
\begin{enumerate}
  \item We establish a~common baseline.
    For~comparing changes in~the performance of~$2$'s root information sets, they need a~common starting point:
    the original strategy $\sigma_1^S$.

    For every $I \in \I_2^{R(S)}$ we subtract the opponent's original counterfactual best response value, setting the utility at~each terminal node~$z \in Z(I)$ to $\ut_2(\zt) = u_2(z) - CBV_2^{\sigma^S_1}(I)$.
    We must not forget to~update $\ut_1(\zt) = -\ut_2(\zt)$ either, as we need the game to remain zero-sum.

    In case of~no deviation from our original strategy $\sigma_1^S$, the shifting gives a~value of~$0$ to~opponent's starting states.
\end{enumerate}
\todo

\begin{figure}[H]
  \centering
  \def\svgwidth{.4\textwidth}
  \input{../img/max-margin-gadget.pdf_tex}
  \def\captionTitle{Our max-margin gadget}
  \caption[\captionTitle]{\captionTitle. When $1$'s original strategy is used, the terminal nodes' offsets enforce the opponent to have a~zero utility in~best response.}
  \label{fig:max-margin-gadget}
\end{figure}

\section{Experimental Results}
\epigraphLong{
  If you find that you're spending almost all your time on theory, start turning some attention to practical things; it will improve your theories. If you find that you're spending almost all your time on practice, start turning some attention to theoretical things; it will improve your practice.
}{Donald Ervin Knuth}
