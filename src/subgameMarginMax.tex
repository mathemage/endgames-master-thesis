\chapter{Subgame-Margin Maximization}
\label{ch:max-margin}
\note{
  This chapter summarizes our own paper~(\cite{Moravcik2016refining}).
}

\section{Motivation}

\section{Subgame Margin}

\begin{defn}[Subgame margin]
\end{defn}

\begin{thm}[improvement is proportional to the subgame margin (and may be greater)]
  \todo
\end{thm}

\begin{proof}
  \todo
\end{proof}

\section{Equivalent Linear Program}

\section{Gadget Game}
\epigraphLong{
  A~new gadget that lasts only five minutes is worth more than an~immortal work that bores everyone.
}{Francis Picabia}
\begin{figure}[H]
  \centering
  \def\svgwidth{.4\textwidth}
  \input{../img/max-margin-gadget.pdf_tex}
  \def\captionTitle{Our max-margin gadget}
  \caption[\captionTitle]{\captionTitle. When $1$'s original strategy is used, the terminal nodes' offsets enforce the opponent to have a~zero utility in~best response.}
  \label{fig:max-margin-gadget}
\end{figure}

\section{Experimental Results}
\epigraphLong{
  If you find that you're spending almost all your time on theory, start turning some attention to practical things; it will improve your theories. If you find that you're spending almost all your time on practice, start turning some attention to theoretical things; it will improve your practice.
}{Donald Ervin Knuth}
