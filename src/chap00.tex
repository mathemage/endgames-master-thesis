\chapter{Background, Definitions and Notations}
\epigraph{
  I can't go to a~restaurant and order food because I keep looking at the fonts on~the menu.
}{Donald Ervin Knuth}

\section{General Game Theory}

\subsection{What Is Game Theory?}

The field of Game Theory deals with interactions or conflicts between $2$ or more agents.
\todo

A~\emph{(simultaneous move) game} consists of: \todo
\begin{itemize}
  \item set $P$ of players $\braces{1, 2, \cdots, n}$
  \item actions
  \item payoffs
\end{itemize}

There are additional concepts related to games such as: \todo
\begin{itemize}
  \item \emph{strategy}
  \item \emph{dominant strategy}
  \item \emph{best response}
  \item \emph{pure Nash equilibrium}
  \item \emph{mixed strategy}
  \item \emph{(mixed) Nash equilibrium}
  \item \emph{correlated equilibrium}
\end{itemize}

A~\emph{game with turns} has:
\begin{itemize}
  \item \todo
\end{itemize}

\subsection{Representation of Games}

\todo
\begin{itemize}
  \item standard (matrix) form
  \item compactly represented game
\end{itemize}

There will be other kinds of representation for extensive form games (see Subsection~\ref{ssec:extensive-form}).

\subsection{Standard Examples}

\todo %Describe the games and how the game-theoretic definitions from above look like in these examples

\begin{itemize}
  \item{Rock-paper-scissors}
  \item{Chess}
  \item{Poker}
\end{itemize}

\section{Combinatorial Game Theory}
\label{sec:CGT}

\todo

\section{The Game of Go}
\label{sec:Go}

\subsection{Rules}

\emph{Black} and \emph{White} place pieces (\emph{stones}) on the unoccupied intersections (\emph{points}) of a~\emph{board} with a~$19\times19$ grid of~lines.
Players take turns, Black moves first.
There are only 2 basic rules of Go:
\begin{description}
  \item [The rule of liberty]
    Every stone remaining on the board must have at least one open point (an~intersection, called a~\emph{liberty}) directly next to it (up, down, left, or right), or must be part of a~connected group that has at least one such liberty next to it.
    \begin{figure}[H]
      \centering
      \includegraphics[width=.5\textwidth]{../img/Go_rule_of_liberty.png}
      \caption{The rule of liberty}
      \label{fig:Go-rule-liberty}
    \end{figure}

    Stones or groups of stones which lose their last liberty are removed from the board.

  \item [The ``ko'' rule]
    The stones on the board must never repeat a~previous position of~stones.
    This is to prevent unending cycles.
    \begin{figure}[H]
      \centering
      \includegraphics[width=.5\textwidth]{../img/Go_ko_rule.png}
      \caption{The ``ko'' rule}
      \label{fig:Go-Ko-rule}
    \end{figure}

\end{description}

\subsection{Scoring, Ranks and Handicaps}

There are several \textbf{scoring rules} to determine the winner of a~game.
In the match of~AlphaGo against Lee Sedol,%
\footnote{See Chapter~\ref{ch:AlphaGo}.}
the \emph{area scoring} was used.
Under area scoring system, player's score is:
\begin{itemize}
  \item the number of stones that the player has on the board
  \item plus the number of~empty intersections surrounded by that player's stones
  \item plus \emph{komi(dashi)} points%
    \footnote{a~compensation for the first move advantage of~the Black player}
    for the White player
\end{itemize}

\emph{Elo rating} can be used to denote players' \textbf{ranks}.
Alternatively, \emph{kyu/dan} (in~Japanese) or \emph{gup/dan} (in~Korean) system is also vastly popular:
\begin{figure}[H]
  \centering
  \includegraphics[width=.8\textwidth]{../img/Go_kyu_dan.png}
  \caption{Kyu/Gup and Dan ranks}
  \label{fig:Go-ranks}
\end{figure}

\textbf{Handicap} system is used to even up differences in ranks:
Black can place 1 or more stones in advance as a~compensation for White's greater strength.

\section{Algorithmic Game Theory}

\subsection{What Is Algorithmic Game Theory?}
One of the classic textbook is the extensive \emph{Algorithmic Game Theory} (\cite{AGT07}).
\todo

\subsection{Examples from Algorithmic Game Theory}

The following games capture various game-theoretic properties and the ``real-life'' counterparts of these games in the fields such as networking, economy etc.
The examples and figures are taken from \emph{Algorithmic Game Theory} (\cite{AGT07}).

\newcommand{\widthratio}{0.3}
\begin{itemize}
  \item \emph{Prisoner's dilemma} \todo
    \begin{figure}[H]
      \centering
      \includegraphics[width=\widthratio\paperwidth]{../img/prisoner.png}
      \caption{Prisoner's dilemma}
      \label{fig:prisoner}
    \end{figure}

    \begin{figure}[H]
      \centering
      \includegraphics[width=\widthratio\paperwidth]{../img/isp.png}
      \caption{ISP routing game}
      \label{fig:isp-routing}
    \end{figure}


  \item \emph{Pollution game} is a multi-player version of Prisoner's dilemma \todo

  \item An~example of a~coordination game is the \emph{Battle of sexes}.

    \begin{figure}[H]
      \centering
      \includegraphics[width=\widthratio\paperwidth]{../img/battle-of-sexes.png}
      \caption{Battle of sexes}
      \label{fig:battle-of-sexes}
    \end{figure}

  \item Another coordination game is the \emph{Routing congestion game}, taken from the world of networking

    \begin{figure}[H]
      \centering
      \includegraphics[width=0.65\paperwidth]{../img/routing-congestion-game.png}
      \caption{Routing congestion game}
      \label{fig:routing-congestion}
    \end{figure}

  \item \emph{Matching pennies} can be considered as a $2$-choice reduction of Rock-paper-scissors.

    \begin{figure}[H]
      \centering
      \includegraphics[width=\widthratio\paperwidth]{../img/matching-pennies.png}
      \caption{Matching pennies}
      \label{fig:matching-pennies}
    \end{figure}

  \item \emph{Pricing game} is an~example of a~game without a~(mixed) Nash equilibrium.

    \begin{figure}[H]
      \centering
      \includegraphics[width=\widthratio\paperwidth]{../img/pricing-game.png}
      \caption{Pricing game}
      \label{fig:pricing-game}
    \end{figure}

  \item \emph{Traffic light}

    \begin{figure}[H]
      \centering
      \includegraphics[width=\widthratio\paperwidth]{../img/traffic-light.png}
      \caption{Traffic light}
      \label{fig:traffic-light}
    \end{figure}

  \item \emph{Ultimatum game}
\end{itemize}
