\chapter{Endgame Solving}
\label{ch:endgame-solving}
\epigraph{
  Poker has the only river in the world you can drown in more than once.
}{an~old Poker joke}
\vskip -2em
\note{
  This chapter summarizes the approach, methods and results of~the authors \cite{GanzfriedSandholm13improving}.
}

\section{Motivation}
\epigraphLong{
  Memorizing a~playbook is like memorizing a~script.
  When they change the script at the last minute it's like changing a~play in~a~game.
}{Michael Strahan}
Two-player zero-sum imperfect-information games can be solved via linear programming (\cite{Koller1994fast}), by modelling sequences of~moves as variables of~a~sequence-form \acrshort{lp} (recall Section~\ref{sec:solving-games-lp}).
This approach scales well to games with up to $10^8$ game states.
Unfortunately, many attractive Poker games are far larger~(\cite{Johanson2013measuring}):
\begin{itemize}
  \item two-player \acrfull{lhe} $\approx 10^{17}$ states
  \item two-player \acrfull{nlhe}\footnotemark{} $\approx 10^{165}$ states.
    \footnotetext{the most popular online variant of~Poker}
\end{itemize}
It is possible to find approximate equilibrium with iterative algorithms, as well.
These methods are guaranteed to converge in~the limit, and scale to~at least $10^{12}$ states (\cite{Hoda2010smoothing}; \cite{Zinkevich2007regret}).
Nevertheless, that is still not enough for the mentioned big poker variants.

Today, a~prevailing approach to enormous imperfect-information games (such as \acrshort{lhe} or \acrshort{nlhe}) is to reduce their sizes by~means of~\emph{abstractions}:
\begin{description}
  \item [Information abstraction] groups together different signals (e.g. similar poker hands).
  \item [Action abstraction] discretizes an~immense action space, making it thus smaller and more manageable.
\end{description}
The method afterwards finds an~\emph{approximate equilibrium} in the abstracted game.

An~appealing idea to diminish harmful effects of~the abstraction and the approximate-equilibrium search, is to~solve endgames dynamically:
an~agent only needs to~deal with those portions of~the game that are actually reached during a~play.
As a~consequence, the endgame can be re-solved with a~finer abstraction and more precise pot and stack sizes.
\todo

\section{Gadget Game}

\section{Equivalent Linear Program}
\todo

The equivalent \acrshort{lp} formulation for the abstracted endgame is the sequence-form \acrshort{lp} (\ref{lp:seq-form}):
\begin{equation*}
  \begin{split}
    \max_{v, x}\  f^\top v & \\
    Ex &= e \\
    F^\top v - A^\top x &\le \vect{0} \\
    x &\ge \vect{0},
  \end{split}
\end{equation*}

\section{Experimental Results}
