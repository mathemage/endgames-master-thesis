\chapter{Setting the Scene for Perfect Information}
\epigraph{
  In order to improve your game, you must study the endgame before everything else.
  For whereas the endings can be studied and mastered by~themselves, the middle game and opening must be studied in relation to the end game.
}{José Raúl Capablanca}

For many games, solving their late stages (so-called \emph{endgames}) can be done in~a~dynamic online way.
In other words, we are often able to postpone the computation of~the endgame strategy until the endgame itself is reached in~the play.

\section{Working Examples}
Subgame solutions are particularly used in~perfect-information games such as chess or Go, where the endgame technique has been used for~long time.
In these domains, online endgame solving has~significant importance, as it substantially improves the playing quality of~agents.

In Chapter~\ref{ch:chess}, we will see the power of~subgame pre-computation with the example of~chess:
solutions to~endings are stored in so-called \emph{endgame tablebase}.
They are used in~real world to~aid professional chess players, either in~proving their guaranteed victory or in~analysing past games.
Moreover, since tablebases are mathematically proven to be~optimal, they provide a~glimpse into the world of~``perfect chess'' played by ``all-knowing super-players''.
