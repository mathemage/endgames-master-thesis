\chapter{Setting the Scene for Perfect Information}
\epigraph{
  In order to improve your game, you must study the endgame before everything else.
  For whereas the endings can be studied and mastered by~themselves, the middle game and opening must be studied in relation to the end game.
}{José Raúl Capablanca}

For many games, solving their late stages (so-called \emph{endgames}) can be done in~a~dynamic online way.
In other words, we are often able to postpone the computation of~the endgame strategy until the endgame itself is reached in~the play.

On the other hand, endgames can be also pre-computed (often up to a~perfect play) and cached for later.
Such a~``memoization'' approach is notably fruitful in~popular games such as Chess.

\section{Extensive Form for Perfect-Information}
\label{sec:extensive-form-perf-info}

Games with perfect information can be naturally represented with a~\emph{game tree}.
Tree representation (rather than a~representation with a~pay-off matrix) is called an extensive form.
Formally, an~\emph{extensive-form game} (\cite[p.~200]{Osborne1994course}) consists of

\begin{itemize}
  \item a finite set of \emph{players} $P$,

  \item a finite set $H$ of all possible \emph{histories},
    \begin{itemize}
      \item Each history consists of individual \emph{actions}.
      \item $h \sqsubseteq h'$ denotes that history~$h$ is a~prefix of $h'$.
      \item $\emptyset \in H$ and $h' \in H \land h \sqsubseteq h' \implies h \in H$
      \item Set $Z \subseteq H$ is the set of \emph{terminal histories}, i.e. histories that are not prefixes of any other histories.
    \end{itemize}

  \item the set of available actions $A(h) = \braces{a: (h, a) \in H}$ for every (non-terminal) node $h \in H \setminus Z$,

  \item a~function $P()$ assigning an~\emph{acting player} to each $h \in H \setminus Z$.
    The acting players are taken from the set $P \cup \braces{c}$, where $c$ is the \textbf{c}hance player (e.~g. a~dice, the card dealer, the nature etc.).
    Thus, $P(h) \in P \cup \braces{c}$ for any $h \in H \setminus Z$.

  \item a \emph{utility function} $u_i\colon Z \goto \R$,

\section{Working Examples}
\epigraph{
  Few things are harder to~put up with than the~annoyance of a~good example.
}{Mark Twain}
Subgame solutions are particularly used in~perfect-information games such as Chess or Go, where the endgame technique has been used for~long time.
In these domains, online endgame solving has~significant importance, as it substantially improves the playing quality of~agents.

In Chapter~\ref{ch:Chess}, we will see the power of~subgame pre-computation with the example of~Chess:
solutions to~endings are stored in so-called \emph{endgame tablebase}.
They are used in~real world to~aid professional Chess players, either in~proving their guaranteed victory or in~analysing past games.
Moreover, since tablebases are mathematically proven to be~optimal, they provide a~glimpse into the world of~``perfect Chess'' played by ``all-knowing super-players''.

In contrast, Chapter~\ref{ch:goCGT} demonstrates how the in-play approach to~endgames is beneficial in~the game of~Go.
Once reaching the late stage, the board is partitioned into distinct parts.
The corresponding (and provably independent) subgames are re-solved ``on the fly'', just to be afterwards combined using the \emph{combinatorial game theory}.

Finally, Chapter~\ref{ch:AlphaGo} reviews the modern approach to~computer Go employed by~Google DeepMind:
the Go program \emph{AlphaGo} combines \emph{Monte Carlo tree search} with \emph{neural networks} to~simulate every play position, as if it were an~endgame.
This in particular means that several moves into the future are simulated and the corresponding part of~the game tree is unrolled and expanded.
The rest of~the tree is discarded, effectively leaving only the relevant subgame for the oncoming play.
