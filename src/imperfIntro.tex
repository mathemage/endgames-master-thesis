\chapter{Setting the Scene for Imperfect Information}
\epigraph{
  Intelligence is a~game of~imperfect information.
  We can guess our opponent's moves, but we can't be sure until the game is over.
}{Khalid Muhammad}

\section{Games with Turns}

\subsection{Extensive Form for Imperfect-Information}
\label{ssec:extensive-form-imperf-info}
\epigraph{
  Education is only a~ladder to~gather fruit from the tree of~knowledge, not the fruit itself.
}{Albert Einstein}
Recall extensive forms (for perfect-information games) from Section~\ref{sec:extensive-form-perf-info}.
An~\emph{extensive form} (for an~imperfect-information game) is the usual extensive form with additional:
\begin{itemize}
  \item \textbf{c}hance player $c$ (e.~g. a~dice, the card dealer, the nature etc.).
    The set of~players is denoted by $P \cup \braces{c}$.
    For the function $P()$ of~acting players we have $P(h) \in P \cup \braces{c}$ for any $h \in H \setminus Z$.
  \item function $f_c$ determining the probability measure over actions $A(h)$ for nodes $h$ with $p(h) = c$, the nodes of the chance player.

  \item partition $\I_i$ of nodes $\braces{h \in H: p(h) = i}$, which is called the \emph{information partition} of player~$i$.
    Its element $I \in \I_i$ is an~\emph{information set} of player~$i$ and $I(h) \in \I_i$ (with $p(h) = i$) denotes the information set containing $h$.

    An information set represents grouping of histories that are indistinguishable from $i$'s point of view.
    In the game of poker, for example, this might be because of hidden cards of opponents.
\end{itemize}

There are further notions related to any extensive-form game:

\begin{itemize}
  \item \emph{Strategy}~$\sigma_i$ of (non-chance) player~$i$ determines a~probability distribution over $A(I)$ at every $I \in \I_i$.
    Thus $\pi ^\sigma (I, a)$ is the probability of action $a$ at the information set~$I$.
    $\Sigma_i$ denotes the set of all possible strategies for player~$i$.
  \item A~\emph{strategy profile} is a~vector of all (non-chance) players' strategies denoted by $\sigma = (\sigma_1, \sigma_2, \ldots, \sigma_ {\abs{P}})$.
    The set of all such possible strategy profiles is denoted by $\Sigma$.
    Hence, it is the Cartesian product $\Sigma = \prod\limits _{i \in P} \Sigma_i$.
  \item We use the Greek letter $\pi$ to evaluate the probability corresponding to a~profile~$\sigma$:
    \[\pi ^\sigma(h) = \prod _{i \in P \cup \braces{c}} \pi _i ^\sigma (h)\]
    
  \item The probability $\pi _{-i} ^\sigma (h)$ (or sometimes just briefly $\pi _{-i} (h)$) is the product of~all players' contribution, except for the one of player~$i$:
    \[\pi _{-i} ^\sigma(h) = \prod _{j \in P \setminus \braces{i} \cup \braces{c}} \pi _j ^\sigma (h)\]
    
  \item $\sigma | _{I \goto a}$ denotes the strategy identical to $\sigma$ with the only one exception:
    the action~$a$ is always played at the information set~$I$.
  \item Given the strategic profile $\sigma$, the \emph{expected utility}~$u_i (\sigma)$ for player~$i$ is defined as:
    \[ u_i (\sigma) = \sum _{z \in Z} \pi ^\sigma (z) u_i(z)\]

  \item A~\emph{best response} $BR _i (\sigma _{-i})$ (or briefly $BR _i (\sigma)$) of player $i$ for given $\sigma _{-p}$ is such a~strategy $\sigma _i \in \Sigma _i$ that maximizes player's expected utility against others:
    \[ u_i (\sigma) = \max _{\sigma'_i \in \Sigma_i} u_i ((\sigma'_i, \sigma_{-i})) \]

  \item A~\emph{Nash equilibrium} (in the context of extensive-form games) is a~strategy profile $\sigma$ such that no player~$i \in P$ has any incentive to deviate from his strategy.
    In other words, all players are playing best responses against each other:
    \[ \forall i \in P\colon u_i (\sigma) = \max _{\sigma'_i \in \Sigma_i} u_i ((\sigma'_i, \sigma_{-i})) \]

  \item The \emph{counterfactual value} $v _i ^\sigma (I)$ is the expected utility provided that the information set $I$ is reached and all players play according to strategy $\sigma$ with exception of player~$i$, who plays to reach $I$:
    \[ v _i ^\sigma (I) = \sum\limits _{h \in I, \; h' \in Z}
      \frac
      {\pi _{-i} ^\sigma(h) \pi ^\sigma(h,h') u_i(h')}
      {\pi _{-i} ^\sigma (I)} \]

  \item A~\emph{counterfactual best response} $CBR _i (\sigma _{-i})$ (or briefly $CBR _i (\sigma)$) of player~$i$ is a~strategy maximizing the counterfactual value at each information set $I \in \I _i$:
    \[ \pi ^\sigma (I, a) \geq 0
      \; \Longleftrightarrow \;
      v _i ^\sigma (I, a) = \max _{a' \in A(I)} v _i ^\sigma (I, a') \]

    Note that $CBR _i (\sigma)$ is always a best response $BR _i (\sigma)$, but the reverse implication does not need to hold:
    a~best response $\sigma$ can select an~arbitrary action in an~unreachable information set $I$ (the one where $\pi ^\sigma (I) = 0$).
    Such best responses are in general not counterfactual best responses.

  \item For the sake of notation's simplicity, we will define \emph{counterfactual best response value} as the counterfactual value for the strategy, where player $i$ plays according $CBR _i (\sigma _{-i})$ rather than the original $\sigma$.
    Formally, it is
    \[ CBV _i ^\sigma (I) = v _i ^{(\sigma _{-i}, CBR _i (\sigma _{-i} ))} (I) \]

\end{itemize}

There may be various properties for extensive-form games:

\begin{itemize}
  \item being \emph{two-player}
  \item having \emph{perfect recall}: any two states from the same information set $I \in \I _i$ share the same history of past actions and player $i$'s information sets.

    In other words, at any stage of the game no player can forget what happened so far:
    neither actions taken nor the information sets reached.
  \item being \emph{zero-sum}: For any $\sigma \in \Sigma$ we have $\sum _{i \in P} u _i (\sigma) = 0$.
\end{itemize}

\subsection{Sequence Form}

\section{Methods of~Solution}

\subsection{Linear Programming}
{
  \setlength{\epigraphwidth}{0.65\textwidth}
  \epigraph{
    If you optimize everything, you will always be unhappy.
  }{Donald Ervin Knuth}
}

\subsection{Learning}
{
  \setlength{\epigraphwidth}{0.65\textwidth}
  \epigraph{
    Perfecting oneself is as~much unlearning as~it is learning.
  }{Edsger Dijkstra}
}

\section{Subgames Revisited}

So far, the information sets have grouped only those states where the player was the acting player.
For the purposes of the subgame, it is also necessary to include the states that are indistinguishable from the point of view of other players.
Therefore, we define the \emph{augmented information sets} (\cite{BurchJohansonBowling13}):
\begin{itemize}
  \item For any player $i \in P$, let $H_i(h)$ be the sequence of player $i$'s information sets reached by player $i$ on the path to $h$, and the actions taken by player~$i$.
  \item An~\emph{augmented information set} $I_i(h)$ is defined by the following characterization:
    for any two states $h, h' \in H$, we have that 
    \[ I_i (h) = I_i (h') \; \Longleftrightarrow \; H_i (h) = H_i (h') \]
\end{itemize}
At this point, we may finally define the notion of a~\emph{subgame}:
\begin{itemize}
  \item An~(imperfect information) \emph{subgame} is a~forest of trees, closed under both the descendant relation and membership within augmented information sets for any player (\cite{BurchJohansonBowling13}).
\end{itemize}

\subsection{Previous Works}
