\chapter{Shortcomings of~the Imperfect Information for Subgames}
{
  \setlength{\epigraphwidth}{0.75\textwidth}
  \epigraph{
    Our shortcomings are the eyes with which we see the ideal.
  }{Friedrich Nietzsche}
}%
In this chapter we will notice how the situation for subgames becomes more delicate within the imperfect-information setting.
A~re-solved subgame strategy can end up being more exploitable:
even if we play a~best response in the subgame, we \textbf{assume the fixed original strategy} in~the trunk of~the game tree.

This is, however, not true since the opponent can distinguish the states within our own information set, which---for us---are indistinguishable.
He can therefore freely change his behavior in~the trunk and manipulate against our best response for his own benefit.

For more illustration, study the example in Section~\ref{sec:intricate-ex}.

\section{The Intricate Example}
\label{sec:intricate-ex}
{
  \setlength{\epigraphwidth}{0.75\textwidth}
  \epigraph{
    It's very simple.
    Scissors cuts paper.
    Paper covers rock.
    Rock crushes lizard.
    Lizard poisons Spock.
    Spock smashes scissors.
    Scissors decapitates lizard.
    Lizard eats paper.
    Paper disproves Spock.
    Spock vaporizes rock.
    And as it always has, rock crushes scissors.
  }{Sheldon Cooper (\emph{The Big Bang Theory}), Season~2, Episode~8}
}%
