\chapter{Shortcomings of~the Imperfect Information for Subgames}
{
  \setlength{\epigraphwidth}{0.75\textwidth}
  \epigraph{
    Our shortcomings are the eyes with which we see the ideal.
  }{Friedrich Nietzsche}
}%
In this chapter we will notice how the situation for subgames becomes more delicate within the imperfect-information setting.
A~re-solved subgame strategy can end up being more exploitable:
even if we play a~best response in the subgame, we \textbf{assume the fixed original strategy} in~the trunk of~the game tree.

This is, however, not true since the opponent can distinguish the states within our own information set, which---for us---are indistinguishable.
He can therefore freely change his behavior in~the trunk and manipulate against our best response for his own benefit.
For more illustration, study the example in Section~\ref{sec:intricate-ex} (taken from \cite{BurchJohansonBowling13}).

\section{The Intricate Example}
\label{sec:intricate-ex}
{
  \setlength{\epigraphwidth}{0.75\textwidth}
  \epigraph{
    It's very simple.
    Scissors cuts paper.
    Paper covers rock.
    Rock crushes lizard.
    Lizard poisons Spock.
    Spock smashes scissors.
    Scissors decapitates lizard.
    Lizard eats paper.
    Paper disproves Spock.
    Spock vaporizes rock.
    And as it always has, rock crushes scissors.
  }{Sheldon Cooper (\emph{The Big Bang Theory}), Season~2, Episode~8}
}%
To see the difficulties of~combining a~re-solved subgame strategy with an~original trunk strategy, we will consider \emph{rock-paper-scissors}\footnotemark{} as our running example:
\footnotetext{a~3-choice extension of~\emph{matching pennies} (Figure~\ref{fig:matching-pennies})}
\begin{figure}[H]
  \centering
  \scriptsize
  \def\svgwidth{.45\textwidth}
  \input{../img/rock-paper-scissors-ext-form.pdf_tex}
  \def\captionTitle{Extensive form of~rock-paper-scissors}
  \caption[\captionTitle]{\captionTitle:\\ Scissors cuts paper. Paper covers rock. Rock crushes scissors. \\(\cite{BurchJohansonBowling13})}
  \label{fig:game-tree-rock-paper-scissors}
\end{figure}

Figure~\ref{fig:game-tree-rock-paper-scissors} displays the extensive form of~rock-paper-scissors:
rather than showing the choices simultaneously, the two players take turns in~picking their action.
However, their decisions are mutually hidden, and revealed only at~the end.

The blue ``bubble'' encircling the 3 acting nodes of player~2 marks his (only) information set $\I_2 = \{R, P, S\}$ (states correspond to the action chosen by~the first player).
This imperfect information represents the fact that he is unaware of~1's preceding choice.

\todo
\begin{figure}[H]
  \centering
  \scriptsize
  \def\svgwidth{.3\textwidth}
  \input{../img/rock-paper-scissors-subgame.pdf_tex}
  \def\captionTitle{An~(imperfect-information) subgame of~rock-paper-scissors}
  \captionWithCite{\captionTitle}{BurchJohansonBowling13}
  \label{fig:subgame-rock-paper-scissors}
\end{figure}
